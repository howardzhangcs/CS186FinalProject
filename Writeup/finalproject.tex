\documentclass[11pt,a4paper]{article}
\usepackage[left=1in,top=1in,bottom=1in,right=1in]{geometry}
\usepackage{enumerate}
\usepackage{amsmath, amsthm}
\usepackage[export]{adjust box}
\usepackage{graphicx}
\graphicspath{ {../Simulation/Graphs/} }
\usepackage[outdir=./]{epstopdf}
\usepackage{tikz}
\usetikzlibrary{shapes.geometric}
\everymath=\expandafter{\the\everymath\displaystyle}
\usepackage[font={small}]{caption}
\begin{document}
\author{Peter Bang, Joon Yang, Howard Zhang}
\title{Computer Science 186: Economics and Computation \\ An Analysis of Automated Market Makers}
\date{\today}
\maketitle\
\section*{Abstract}
\section*{Introduction}
Market makers serve an important function in ... \\ \\
In a continuous double auction, the exchange takes on no risk, but liquidity in the market may be
limited and transaction costs may be high. In class, we looked at automated market makers, which
are provided by the exchange (e.g. prediction market) and has an attractive property of a bounded
loss. \\ \\ 
We were interested in exploring in more detail recent developments in automated market makers.  This project was motivated by the Logarithmic Market Scoring Rules (LMSR) (Hanson, 2002) as well as developments in the fields of financial economics and market microstructure. Specifically, we look at the liquidity sensitive LMSR (Othman et al., 2010), and the ...
\section*{Simulation}
\section*{Conclusion}
\section*{Appendix}
\section*{Sources}
\section*{Contribution and Thanks}
\end{document}